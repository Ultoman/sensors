\newpage
\section{Sub-packets from Coresense}

As shortly explained in document section \ref{ssec:sub-pack}, data sub-packets from coresense are generated depending on its data reading from each sensor if valid. The first byte of the sub-packet from coresense is sensor ID for each parameter, and the second byte means validity of the packet and length of the sensor data as shown in Table \ref{table:packsegments}. Detail of sub-packet and sensor data will be explined in this section.


\subsection{Parameters}

The sensor boards output a set of parameters which are identified by a unique ID. Each parameter
has a set of values associated with it which are encoded in an appropriate data format. The table
below lists the various parameters produced by the sensor boards, the unique source ID used to identify them, the values produced by them.


\begin{center}
\begin{longtable}{|l|c|>{\centering}p{0.3\textwidth}|c|}
\caption{Data sub-packet structure (each row is a "chunk")} \label{tab:dataChunk} \\

\hline \rowcolor{black!8} \multicolumn{1}{|c|}{\textbf{Parameter}} & \multicolumn{1}{c|}{\textbf{Source ID}} & \multicolumn{1}{c|}{\textbf{Values}} & \multicolumn{1}{c|}{\textbf{Length}} \\ \hline
\endfirsthead

\multicolumn{4}{c}%
{{\bfseries \tablename \thetable{} -- continued from previous page}} \\
\hline \rowcolor{black!8} \multicolumn{1}{|c|}{\textbf{Parameter}} & \multicolumn{1}{c|}{\textbf{Source ID}} & \multicolumn{1}{c|}{\textbf{Values}} & \multicolumn{1}{c|}{\textbf{Length}} \\ \hline 
\endhead

\rowcolor{black!8} \multicolumn{4}{|r|}{{Continued on next page}} \\ \hline
\endfoot

\hline
\endlastfoot

        \multirow{3}{*}{Firmware version} & \multirow{3}{*}{0xFf} & Firmware version (HW/SW) & 2 bytes \\ \cline{3-4}
        & & Build time & 4 bytes \\ \cline{3-4}
        & & Build git & 2 bytes \\ \hline

    \rowcolor{black!5} \multicolumn{4}{|c|}{{Metsense board}} \\ \hline
        Metsense/Lightsense MAC address & 0x00 & MAC Address & 6 bytes \\ \hline
        TMP112 & 0x01 & Temperature & 2 bytes \\ \hline
        \multirow{2}{*}{HTU21D} & \multirow{2}{*}{0x02} & Temperature & 2 bytes \\ \cline{3-4}
        & & relative humidity & 2 bytes \\ \hline
        hih4030 & 0x03 & Humidity & 2 bytes \\ \hline
        \multirow{2}{*}{BMP180} & \multirow{2}{*}{0x04} & Temperature & 2 bytes \\ \cline{3-4}
        & & Pressure & 3 bytes  \\ \hline
        PR103J2 & 0x05 & Temperature & 2 bytes \\ \hline
        TSL250RD & 0x06 & Visible Light & 2 bytes \\ \hline
        \multirow{4}{*}{MMA8452Q} & \multirow{4}{*}{0x07} & Acceleration in X & 2 bytes \\ \cline{3-4}
        & & Acceleration in Y & 2 bytes \\ \cline{3-4}
        & & Acceleration in Z & 2 bytes \\ \cline{3-4}
        SPV1840LR5H-B & 0x08 & RMS Sound Level & 128 bytes \\ \hline
        TSYS01 & 0x09 & Temperature & 2 bytes \\ \hline
        
    \rowcolor{black!8} \multicolumn{4}{|c|}{{Lightsense board}} \\ \hline
        \multirow{3}{*}{HMC5883L} & \multirow{3}{*}{0x0A} & Magnetic Field in Z & 2 bytes \\ \cline{3-4}
        & & Magnetic Field in Y & 2 bytes \\ \cline{3-4}
        & & Magnetic Field in Z & 2 bytes \\ \hline
        \multirow{2}{*}{HIH6130} & \multirow{2}{*}{0x0B} & Temperature & 2 bytes \\ \cline{3-4}
        & & relative humidity & 2 bytes \\ \hline
        APDS-9006-020 & 0x0C & Ambient light intensity & 3 bytes \\ \hline
        TSL260RD & 0x0D & IR intensity & 3 bytes \\ \hline
        TSL250RD & 0x0E & Visible light intensity & 3 bytes \\ \hline
        MLX75305 & 0x0F & Light & 3 bytes \\ \hline 
        ML8511 & 0x10 & UV intensity & 3 bytes \\ \hline
        TMP421 & 0x13 & Temperature & 2 bytes \\ \hline

    \rowcolor{black!8} \multicolumn{4}{|c|}{{Chemsense board}} \\ \hline
        Chemsense configuration & 0x16 & Chemsense FW config & 1514 bytes \\ \hline
        Chemsense reading & 0x2A & Raw Reading & Varies \\ \hline

     \rowcolor{black!8} \multicolumn{4}{|c|}{{Alpha Sensor}} \\ \hline
        \multirow{9}{*}{Histogram} & \multirow{9}{*}{0x28} & Bin count & 32 bytes \\ \cline{3-4}
        & & Average Time & 4 bytes \\ \cline{3-4}
        & & Sample flow rate & 4 bytes \\ \cline{3-4}
        & & Temp/Pressure(alter) & 4 bytes\\ \cline{3-4}
        & & Sampling period & 4 bytes \\ \cline{3-4}
        & & Sum of the counts & 2 bytes \\ \cline{3-4}
        & & PM 1 & 4 bytes \\ \cline{3-4}
        & & PM 2.5 & 4 bytes \\ \cline{3-4}
        & & PM 10 & 4 bytes \\ \hline
        Serial & 0x29 & Serial Number & 20 bytes \\ \hline
        Firmware & 0x30 & Firmware version & 2 bytes \\ \hline
        \multirow{9}{*}{Configuration} & \multirow{9}{*}{0x31} & Bin Boundaries & 32 bytes \\ \cline{3-4}
        & & Bin Particle Volumes & 64 bytes \\ \cline{3-4}
        & & Bin Particle Densities & 64 bytes \\ \cline{3-4}
        & & Bin Sample Volume Weightings & 64 bytes \\ \cline{3-4}
        & & Gain Scaling Coefficient & 4 bytes \\ \cline{3-4}
        & & Sample Flow Rate & 4 bytes \\ \cline{3-4}
        & & Laser DAC & 1 byte \\ \cline{3-4}
        & & Fan DAC & 1 byte \\ \cline{3-4}
        & & Conversion factor & 1 byte \\ \cline{3-4}
        & & Space Bytes & \\ 

\end{longtable}
\end{center}


\newpage
\subsection{Data packets}
The context of each parameter, its utility and the arrangement of its values is described below. In all
the tables below, the validity bit is set to 1, which means the data is valid. The parameter described
below are aggregated based on the sensor-board they are situated on -
Metsense, Lightsense and Chemsense.

\subsubsection{Firmware Version}
This is a 8 bytes version information that identifies hardware version, software version, and build information of the waggle node.
The build time and the build git are included to varify the effectiveness of the software.
Firmware version is bit masked and encoded through format 1, and build git is encoded through format 1.

\begin{table}[h!]
    \centering
    \caption{Sub-packet of Firmware version}
    \begin{tabular}{|c|c|c|c|c|}
        \hline
        \rowcolor{black!8}
        \textbf{0xFD} & \textbf{0x88} & \textbf{Firmware version} & \textbf{Build time} & \textbf{Build git} \\ \hline
        Byte[0] & Byte[1] & Bytes[2 -- 3] & Bytes[4 -- 7] & Bytes[8 -- 9]\\ \hline
    \end{tabular}
\end{table}


\newcolumntype{a}{>{\columncolor{black!8}}c}
\begin{table}[h!]
    \centering
    \caption{Firmware version}
    \begin{tabular}{|a|c|}
        \hline
        \textbf{3 bit major HW ver. | 3 bit minor HW ver. | 2 bit major SW ver.} & Byte[2] \\ \hline
        \textbf{2 bit major SW ver. | minor SW ver. $\times$ 10 + sub SW ver.} & Byte[3]\\ \hline
    \end{tabular}
\end{table}
